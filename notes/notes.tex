\documentclass[12pt]{article}
\usepackage{amsmath, amssymb, amsfonts, amsthm}
\usepackage{graphicx}
\usepackage{tikz}
\usepackage{enumitem}
\usepackage{xcolor}
\usepackage{dsfont}
\usepackage{hyperref}
\usepackage{cleveref}
\usetikzlibrary{patterns.meta}

\title{Digital signal processing}
\author{}

%math operators%
\newcommand{\Jac}{J}

\begin{document}

\newcommand{\todo}[1]{\begingroup \color{red} #1 \endgroup}

%common sets%
\newcommand{\R}{\mathbb{R}}
\newcommand{\C}{\mathbb{C}}
\newcommand{\N}{\mathbb{N}}
\newcommand{\Z}{\mathbb{Z}}
\newcommand{\Q}{\mathbb{Q}}

\newcommand{\curlyA}{\mathcal{A}}
\newcommand{\curlyB}{\mathcal{B}}
\newcommand{\curlyC}{\mathcal{C}}
\newcommand{\curlyP}{\mathcal{P}}
\newcommand{\curlyL}{\mathcal{L}}

%linear algebra%
\newcommand{\innerpr}[2]{\left\langle #1, #2 \right\rangle}
\newcommand{\norm}[1]{\left\lVert#1\right\rVert}
\newcommand{\spanof}{\operatorname{span}}
\newcommand{\rank}{\operatorname{rank}}
\newcommand{\tr}{\operatorname{tr}}
\newcommand{\Ker}{\operatorname{Ker}}
\newcommand{\Img}{\operatorname{Im}}
\newcommand{\mat}[1]{\begin{pmatrix} #1 \end{pmatrix}}

%other%
\newcommand{\ind}{\mathds{1}}
\newcommand{\dx}[1]{\, \mathrm{d}#1}

%blocks%
\newtheorem{theorem}{Theorem}[section]
\newtheorem{lemma}{Lemma}[section]

\maketitle
\tableofcontents

\newpage
\section{Conventions}
Let $f:\mathbb{R} \to \mathbb{C}$ be a $2\pi$-periodic, integrable function.
The Fourier coefficients of $f$ are defined by
\[
    \hat f(k)
    =
    \frac{1}{2\pi}\int_0^{2\pi} f(x)\,e^{-ikx} \dx{x},
    \quad k \in \mathbb{Z}.
\]

The Fourier series of $f$ is given by
\[
    f(x)
    \;=\;
    \sum_{k\in\mathbb{Z}} \hat f(k)\,e^{ikx}
\]

For $2\pi$-periodic functions $f,g \colon \mathbb{R}\to\mathbb{C}$ we define the
circular convolution by
\[
    (f * g)(x)
    \;=\;
    \int_0^{2\pi} f(y)\,g(x-y) \dx{y}
\]

The root of unity:
\[\omega_N = e^{\frac{2\pi i}{N}}\]

\newpage
\section{DFT}
\begin{lemma} The following holds for roots of unity:
    \begin{enumerate}[label=\thelemma.\arabic*]
        \item $\omega_N^{kN} = 1 \; \forall{k\in \Z}$ \label{lm:roots of unity cyclicity}
        \item $N$ even. $\omega_{N/2} = \omega_N^2$ \label{lm:roots of unity fft}
        \item $\overline{\omega_{N}} = \omega_N^{-1}$ \label{lm:roots of unity conjugate}
    \end{enumerate}
\end{lemma}
\begin{proof} \hfill
    \begin{enumerate}
        \item
              \[\omega_N^N
                  = {\left(e^{\frac{2\pi i}{N}}\right)}^N
                  = e^{\frac{2\pi iN}{N}}
                  = e^{2\pi i}
                  = \cos(2\pi) + i \sin(2\pi) = 1\]

              Then:
              \[\omega_N^{kN}
                  = {\left(\omega_N^{N}\right)}^k = 1^k = 1\]

        \item $\omega_{N/2}
                  = e^{\frac{2\pi i}{N/2}}
                  = e^{2\cdot \frac{2\pi i}{N}}
                  = {\left(e^{\frac{2\pi i}{N}}\right)}^2
                  = \omega_N^2$
        \item $\overline{\omega_{N}}
                  = \overline{e^{\frac{2\pi i}{N}}}
                  = e^{-\frac{2\pi i}{N}}
                      = {\left(e^{\frac{2\pi i}{N}}\right)}^{-1}
                  = \omega_N^{-1}$
    \end{enumerate}
\end{proof}

Suppose we have $f: \mathbb{R} \rightarrow \mathbb{C}$ which is $2\pi$-periodic.
Consider the values of this function at points $\frac{0\cdot 2\pi}{N}$,  $\frac{1\cdot 2\pi}{N}$, ..., $\frac{(N-1)\cdot 2\pi}{N}$:
\[x_j:= \frac{j\cdot 2\pi}{N}\]
\[y_j := f(x_j)
    = \sum_{k=-\infty}^\infty \hat{f}(k)e^{ikx_j}
    = \sum_{k=-\infty}^\infty \hat{f}(k)e^{2\pi i\frac{kj}{N}}
    = \sum_{k=-\infty}^\infty \hat{f}(k)\omega_N^{kj}\]

By Lemma~\ref{lm:roots of unity cyclicity} we have that
$\omega_N^{(k+nN)j}
    = \omega_N^{kj}\omega_N^{(nj)N}
    = \omega_N^{kj}$

By grouping the terms with $k_1 = k_2 \mod N$ and obtain the following:

\begin{equation}
    y_j
    = \sum_{k = 0}^{N-1} \omega_N^{kj} \left(\sum_{n=-\infty}^{\infty} \hat{f}(k+nN) \right)
    = \sum_{k =0}^{N-1} \omega_N^{kj} z_k
    \label{eq:IDFT unnormalized}
\end{equation}

We define the matrix $F_{kj} = \frac{1}{\sqrt N} \omega_N^{kj}$. Then:
\begin{equation}
    y_j = \sqrt{N}{(Fz)}_j \Rightarrow
    y = \sqrt{N}Fz \Rightarrow
    z = \frac{1}{\sqrt{N}}F^{-1}y
    \label{eq:DFT unnormalized}
\end{equation}

Later we use the vector $z$ to indicate the output of the DFT, namely $F^{-1}y$

The crucial observation is that the oscillation of the form $e^{i (k+nN)x}$ for $n>0$ and $k \in {0, ..., N-1}$ cannot be detected using the given discretized points. This simply follows from the fact that
$e^{i (k+nN)x_j}
    = e^{i (k+nN)\frac{j\cdot 2\pi}{N}}
    = e^{i\frac{kj\cdot 2\pi}{N}}e^{2\pi i\cdot nj}
    = e^{2\pi i\frac{kj}{N}}
    = e^{ikx_j} $ as computed previously. Even though the original signal might have an oscillation with rate above $N$, we are unable to detect it using the discrete values of the function $f$, therefore we may just assume each $z_k$ corresponds to the coefficient $\hat{f}(k)$ (basically, contribution of the waves corresponding to the frequencies above $N$ cannot be distinguished from the wave with the respective frequency in the range from $0$ to $N-1$). In sound processing this doesn't pose any issue since human ear cannot hear any frequency above 20 khz, so with enough discretization we basically do not lose any perceivable data.

\begin{theorem} \label{thm:DFT unitary}
    The matrix $F_{kj} = \frac{1}{\sqrt N} \omega_N^{kj}$ is unitary
\end{theorem}
\begin{proof}
    \[\begin{aligned}
            {(F \overline{F^T})}_{jl}
            = \frac{1}{\sqrt N}\overline{\left(\frac{1}{\sqrt N}\right)}\sum_{k=0}^{N-1}F_{jk}{(\overline{F^T})}_{kl}
            = \frac{1}{N} \sum_{k=0}^{N-1}F_{jk}\overline{F_{lk}}
             & = \\ \frac{1}{N} \sum_{k=0}^{N-1}\omega_N^{jk}\overline{\omega_N^{lk}}
            \underset{(a)}{=} \frac{1}{N} \sum_{k=0}^{N-1}\omega_N^{jk}\omega_N^{-lk}
            = \frac{1}{N} \sum_{k=0}^{N-1}\omega_N^{(j-l)k}
        \end{aligned}\]

    In (a) we used the identity $\overline{\omega_N^{lk}} = \omega_N^{-lk}$ from Lemma~\ref{lm:roots of unity conjugate}

    \begin{enumerate}
        \item $j \neq l$. Using the formula for sum of geometric series:
              \[
                  \sum_{k=0}^{N-1}\omega_N^{(j-l)k}
                  = \frac{1 - {\left(\omega_N^{j-l}\right)}^N}{1 - \omega^{j-l}}
                  = \frac{1 - {\left(\omega_N^{N}\right)}^{j-l}}{1 - \omega^{j-l}}
                  \underset{(a)}{=} \frac{1 - 1^{j-l}}{1 - \omega^{j-l}}
                  = 0\]
              In (a) we used the identity $\omega_N^{N} = 1$ from Lemma~\ref{lm:roots of unity cyclicity}.
              The condition $j \neq l$ ensures that the denominator is non-zero, so the expression is well-defined.
        \item $j = l$. Then:
              \[\sum_{k=0}^{N-1}\omega_N^{(j-l)k}
                  = \sum_{k=0}^{N-1}1
                  = N \Rightarrow
                  {(F \overline{F^T})}_{jl} = \frac{1}{N}N=1\]
    \end{enumerate}

    The matrix $FF^* = I$, so $F$ unitary and $F^{-1}=\overline{F^T}$

    The entries of $F^{-1}$ are given by
    $F^{-1}_{ij} = (\overline{F_{ji}}) 
    = \frac{1}{\sqrt{N}}\overline{\left(\omega_N^{ji}\right)} 
    = \frac{1}{\sqrt{N}}\omega_N^{-ij}$.
\end{proof}

Using the result of Theorem~\ref{thm:DFT unitary} and equations \eqref{eq:DFT unnormalized}, \eqref{eq:IDFT unnormalized} we define the discrete Fourier transform:
\begin{equation}
    z_j
    = {(F^{-1}y)}_j
    = \frac{1}{\sqrt{N}}\sum_{k =0}^{N-1} \omega_N^{-kj} y_k
    \label{eq:DFT} \tag{DFT}
\end{equation}
\begin{equation}
    y_j
    = {(Fz)}_j
    = \frac{1}{\sqrt{N}}\sum_{k =0}^{N-1} \omega_N^{kj} z_k
    \label{eq:IDFT} \tag{IDFT}
\end{equation}

\begin{theorem}[Real signal conjugate symmetry] \label{thm: real signal symmetry}
    Let $N$ be even. The following are equivalent:

    \begin{enumerate}
        \item $y = \overline{y}, y\in \C^N$
        \item $z_{j} = \overline{z_{N-j}} \quad \forall j \in \{1, \dots, N/2-1\}$ and $z_0, z_{N/2} \in \R$
    \end{enumerate}
\end{theorem}
\begin{proof} \hfill \break
    $"(1)\Rightarrow (2)"$ Using the formula for \ref{eq:DFT}
    \[\begin{aligned}
            z_j
            = \frac{1}{\sqrt N} \sum_{k=0}^{N-1} \omega_N^{-kj}y_k
            = \frac{1}{\sqrt N} \sum_{k=0}^{N-1} \overline{\omega_N^{kj}} y_k
            \underset{y=\overline{y}}{=} \frac{1}{\sqrt N} \sum_{k=0}^{N-1} \overline{\omega_N^{kj}y_k}
             & = \\ \overline{ \left(\frac{1}{\sqrt N} \sum_{k=0}^{N-1} \omega_N^{kj} y_k \right)}
            \underset{\omega_N^{kN}=1}{=} \overline{ \left(\frac{1}{\sqrt N} \sum_{k=0}^{N-1} \omega_N^{-kN}\omega_N^{kj} y_k \right)}
             & = \\ \overline{ \left(\frac{1}{\sqrt N} \sum_{k=0}^{N-1} \omega_N^{-k(N-j)} y_k \right)}
            = \overline{z_{N-j}}
        \end{aligned}\]
    \[z_0
        = \frac{1}{\sqrt N} \sum_{k=0}^{N-1} \omega_N^{-k\cdot 0}y_k
        = \frac{1}{\sqrt N} \sum_{k=0}^{N-1} y_k \in \R\]
    Observe:
    \[\omega_N^{-k\cdot N/2}
        = {(e^{-2\pi i \frac{N/2}{N}})}^k
        = {(e^{-\pi i})}^k = {(-1)}^k\]
    \[z_{N/2}
        = \frac{1}{\sqrt N} \sum_{k=0}^{N-1} \omega_N^{-k\cdot N/2}y_k
        = \frac{1}{\sqrt N} \sum_{k=0}^{N-1} {(-1)}^k y_k \in \R\]

    \noindent $"(2)\Rightarrow (1)"$ Using the formula for \ref{eq:IDFT}
    \begin{equation}
        \begin{aligned}
            y_j
            = \frac{1}{\sqrt{N}}\sum_{k =0}^{N-1} \omega_N^{kj} z_k 
            = \frac{1}{\sqrt{N}} \left(z_0 + {(-1)}^jz_{N/2} + \sum_{k = 1}^{N/2-1} \omega_N^{kj} z_k + \sum_{l = N/2+1}^{N-1} \omega_N^{lj} z_l\right)
        \end{aligned} \label{eq: real signal symmetry inv 1}
    \end{equation}
    \[\begin{aligned}
            \sum_{k = 1}^{N/2-1} \omega_N^{kj} z_k + \sum_{l = N/2+1}^{N-1} \omega_N^{lj} z_l
            \underset{l'=N-l}{=} \sum_{k = 1}^{N/2-1} \omega_N^{kj} z_k + \sum_{l = 1}^{N/2-1} \omega_N^{(N-l)j} z_{N-l}
             & = \\ \sum_{k = 1}^{N/2-1} \left(\omega_N^{kj} z_k + \omega_N^{(N-k)j} z_{N-k}\right)
            = \sum_{k = 1}^{N/2-1} \left(\omega_N^{kj} z_k + \overline{\omega_N^{kj}z_k}\right) \in \R
        \end{aligned}\]
    By plugging this expression back to \eqref{eq: real signal symmetry inv 1} we get:
    \[\eqref{eq: real signal symmetry inv 1}
    = \frac{1}{\sqrt{N}} \left(z_0 + {(-1)}^jz_{N/2} + \sum_{k = 1}^{N/2-1} \left(\omega_N^{kj} z_k + \overline{\omega_N^{kj}z_k}\right)\right) \in \R\]
    since all the summands and the factor in front are real.
\end{proof}

\newpage
\section{FFT}
\end{document}